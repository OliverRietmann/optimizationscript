\section{Optimization} \label{sec:factorial}
\begin{tcolorbox}
	Die Fakultät einer Zahl $n\in\mathbb N_0$ is definiert als
	\begin{equation*}
		n!=1\cdot 2\cdot 3\cdot\ldots\cdot \left(n-1\right)\cdot n,
	\end{equation*}
	wobei $0!=1$ und $1!=1$.
	Sind $n$ Elemente auf $n$ Plätze zu verteilen, so gibt es dafür $n!$ Möglichkeiten.
\end{tcolorbox}
\begin{example}
	Hier sind zwei Beispiele:
	\begin{tasks}(2)
		\task $3!=1\cdot 2\cdot 3=6$
		\task $\dfrac{5!}{3!}=\dfrac{1\cdot 2\cdot 3\cdot 4\cdot 5}{1\cdot 2\cdot 3}=4\cdot 5=20$
	\end{tasks}
\end{example}
\begin{exercise}
	Berechnen Sie folgende Ausdrücke.
	Nehmen Sie dabei $n\geq 2$ an.
	\begin{tasks}(4)
		\task $4!$
		\task $\dfrac{6!}{3!}$
		\task $\dfrac{\left(n+1\right)!}{\left(n-1\right)!}$
		\task $\dfrac{n!}{n\cdot\left(n-1\right)}$
	\end{tasks}
\end{exercise}
\begin{comment}
\begin{solution*}
	\phantom{text}
	\begin{tasks}(4)
		\task $24$
		\task $120$
		\task $n\cdot\left(n+1\right)$
		\task $\left(n-2\right)!$
	\end{tasks}
\end{solution*}
\end{comment}
\begin{exercise}
	Wir stellen 4 Leute in eine Reihe.
	Wie viele Reihenfolgen sind möglich?
\end{exercise}
\begin{comment}
	\begin{solution*}
		Es gibt $4!$ Möglichkeiten.
	\end{solution*}
\end{comment}
\begin{exercise}
	Auf einem Parkplatz sind noch 6 Parkplätze frei.
	Gleichzeitig kommen\ldots
	\begin{tasks}(3)
		\task \ldots $3$ Autos an.
		\task \ldots $5$ Autos an.
		\task \ldots $6$ Autos an.
	\end{tasks}
	Wie viele Möglichkeiten gibt es, die freien Parkplätze den ankommenden Autos zuzuteilen?
	Schreibe die Lösung als Division von zwei Fakultäten.
\end{exercise}
\begin{comment}
\begin{solution*}
	\phantom{text}
	\begin{tasks}(3)
		\task $\dfrac{6!}{3!}=120$
		\task $\dfrac{6!}{1!}=720$
		\task $\dfrac{6!}{0!}=720$
	\end{tasks}
\end{solution*}
\end{comment}
\begin{exercise}
	Auf einem Parkplatz mit $n$ Plätzen wollen $k$ Autos parkieren.
	Wie viele Möglichkeiten gibt es, die freien Parkplätze den ankommenden Autos zuzuteilen? Nehmen Sie $k\leq n$ an.
\end{exercise}
\begin{comment}
\begin{solution*}
	\phantom{text}
	Es gibt $\dfrac{n!}{\left(n-k\right)!}$ Möglichkeiten.
\end{solution*}
\end{comment}
\begin{exercise}
	In einem französischen Restaurant werden bei einem Menü mit 4 Gängen folgende Auswahlmöglichkeiten angeboten:
	Zuerst entweder ein Salat oder eine Suppe, anschliessend 4 verschiedene Vorspeisen, 3 Hauptspeisen und 5 Desserts.
	Wie viele Bestellmöglichkeiten gibt es, wenn kein Gang ausgelassen wird?
\end{exercise}
\begin{comment}
\begin{solution*}
	Tatsächlich braucht man hier keine Fakultät.
	Die Anzahl der Bestellmöglchkeiten ist
	\begin{equation*}
		2\cdot 4\cdot 3\cdot 5=120.
	\end{equation*}
\end{solution*}
\end{comment}